\documentclass[conference]{IEEEtran}
\IEEEoverridecommandlockouts
% The preceding line is only needed to identify funding in the first footnote. If that is unneeded, please comment it out.
\usepackage{cite}
\usepackage{amsmath,amssymb,amsfonts}
\usepackage{algorithmic}
\usepackage{graphicx}
\usepackage{textcomp}
\usepackage{listings}
\usepackage{xcolor}

\lstset{
    language=Python,
    basicstyle=\ttfamily\footnotesize,
    keywordstyle=\color{blue},
    stringstyle=\color{red},
    commentstyle=\color{green!60!black},
    numbers=left,
    numberstyle=\tiny,
    stepnumber=1,
    numbersep=5pt,
    backgroundcolor=\color{white},
    showspaces=false,
    showstringspaces=false,
    showtabs=false,
    frame=single,
    tabsize=2,
    captionpos=b,
    breaklines=true,
    breakatwhitespace=false,
    escapeinside={\%*}{*)}
}

\begin{document}

\title{Simulação do controle FOC de motor PMSM aplicado a um motor BLDC\\
}

\author{\IEEEauthorblockN{1\textsuperscript{st} Felipe Lenschow}
\IEEEauthorblockA{\textit{Programa de pós graduação em engenharia elétrica} \\
\textit{Universidade do Estado de Santa Catarina}\\
Joinville, Santa Catarina, Brasil \\
felipe.lenschow@edu.udesc.br}
}

\maketitle

\begin{abstract}
Este artigo apresenta a modelagem e simulação de um sistema de acionamento de Motor Síncrono de Ímãs Permanentes (PMSM) utilizando Controle Orientado a Campo (FOC). O modelo matemático do PMSM no referencial dq é derivado, e uma estratégia de controle empregando controladores Proporcional-Integral (PI) para regulação de velocidade e corrente é implementada. A simulação é desenvolvida em Python, permitindo uma análise modular e flexível do comportamento dinâmico do motor sob condições variadas de carga e velocidade. Os resultados demonstram a eficácia da estratégia FOC em manter um controle preciso de velocidade e geração eficiente de torque.
\end{abstract}

\begin{IEEEkeywords}
PMSM, Controle Orientado a Campo, Simulação, Python, Acionamento de Motor
\end{IEEEkeywords}

\section{Introdução}
Motores Síncronos de Ímãs Permanentes (PMSMs) são amplamente utilizados em aplicações industriais, veículos elétricos e robótica devido à sua alta eficiência, alta densidade de potência e excelente desempenho dinâmico. Para alcançar um controle de alto desempenho, o Controle Orientado a Campo (FOC) é comumente empregado. O FOC permite o controle independente de fluxo e torque transformando as correntes trifásicas do estator para um referencial girante (referencial dq) alinhado com o fluxo do rotor \cite{b1}.

Este artigo detalha o desenvolvimento de um ambiente de simulação para um sistema de acionamento PMSM. A simulação inclui a física do motor, o inversor de fonte de tensão e o algoritmo FOC. O objetivo é fornecer uma compreensão clara da dinâmica do sistema e validar a estratégia de controle através de simulação numérica.

\section{Modelo do Sistema}

\subsection{Modelo Matemático do BLDC}
O modelo dinâmico do motor BLDC pode ser derivado a partir das equações de tensão de fase. Conforme descrito em \cite{b2}, as tensões nos enrolamentos do estator são definidas por:

\begin{equation}
    \begin{bmatrix} v_a \\ v_b \\ v_c \end{bmatrix} = 
    \begin{bmatrix} R_s & 0 & 0 \\ 0 & R_s & 0 \\ 0 & 0 & R_s \end{bmatrix} 
    \begin{bmatrix} i_a \\ i_b \\ i_c \end{bmatrix} + 
    \frac{d}{dt} \begin{bmatrix} \psi_a \\ \psi_b \\ \psi_c \end{bmatrix}
    \label{eq:vabc}
\end{equation}

onde $\psi$ representa o fluxo total concatenado em cada enrolamento, dado por:

\begin{equation}
    \begin{bmatrix} \psi_a \\ \psi_b \\ \psi_c \end{bmatrix} = 
    \begin{bmatrix}
    L_{aa} & L_{ab} & L_{ac} \\
    L_{ba} & L_{bb} & L_{bc} \\
    L_{ca} & L_{cb} & L_{cc}
    \end{bmatrix}
    \begin{bmatrix} i_a \\ i_b \\ i_c \end{bmatrix} + 
    \begin{bmatrix} \psi_{am} \\ \psi_{bm} \\ \psi_{cm} \end{bmatrix}
\end{equation}

A matriz de indutâncias $\mathbf{L}_{abc}$ contém termos que variam com a posição do rotor $\theta_m$ devido à saliência dos polos. As indutâncias próprias e mútuas são dadas por:

\begin{align}
    L_{aa} &= L_{al} + L_{aa0} + L_{g2} \cos(2\theta_e) \\
    L_{bb} &= L_{al} + L_{aa0} + L_{g2} \cos(2\theta_e + 2\pi/3) \\
    L_{cc} &= L_{al} + L_{aa0} + L_{g2} \cos(2\theta_e - 2\pi/3) \\
    L_{ab} &= -\frac{1}{2}L_{aa0} + L_{g2} \cos(2\theta_e - 2\pi/3) \\
    L_{bc} &= -\frac{1}{2}L_{aa0} + L_{g2} \cos(2\theta_e) \\
    L_{ca} &= -\frac{1}{2}L_{aa0} + L_{g2} \cos(2\theta_e + 2\pi/3)
\end{align}

onde $L_{al}$ é a indutância de dispersão, $L_{aa0}$ é a componente constante da indutância mútua e $L_{g2}$ representa a amplitude da variação de indutância devido à saliência.

Para simplificar a análise, aplica-se a Transformada de Park para converter as variáveis do referencial trifásico ($abc$) para o referencial síncrono girante ($dq0$). A transformação é definida por $\mathbf{x}_{dq0} = \mathbf{T} \mathbf{x}_{abc}$, onde $\mathbf{T}$ é a matriz de transformação dada por:

\begin{equation}
    \mathbf{T} = \frac{2}{3} \begin{bmatrix} 
    \cos(\theta_e) & \cos(\theta_e - \frac{2\pi}{3}) & \cos(\theta_e + \frac{2\pi}{3}) \\
    -\sin(\theta_e) & -\sin(\theta_e - \frac{2\pi}{3}) & -\sin(\theta_e + \frac{2\pi}{3}) \\
    \frac{1}{2} & \frac{1}{2} & \frac{1}{2}
    \end{bmatrix}
\end{equation}

Aplicando a Transformada de Park na equação \eqref{eq:vabc}, obtemos a equação de tensão no referencial $dq0$:

\begin{equation}
    \mathbf{v}_{dq0} = \mathbf{T} \mathbf{v}_{abc} = \mathbf{T} \mathbf{R} \mathbf{i}_{abc} + \mathbf{T} \frac{d\boldsymbol{\psi}_{abc}}{dt}
\end{equation}

Sabendo que $\mathbf{i}_{abc} = \mathbf{T}^{-1} \mathbf{i}_{dq0}$ e $\boldsymbol{\psi}_{abc} = \mathbf{T}^{-1} \boldsymbol{\psi}_{dq0}$, e assumindo $\mathbf{R}$ diagonal e constante:

\begin{equation}
    \mathbf{v}_{dq0} = \mathbf{R} \mathbf{i}_{dq0} + \mathbf{T} \frac{d}{dt} (\mathbf{T}^{-1} \boldsymbol{\psi}_{dq0})
\end{equation}

Expandindo a derivada do produto $\frac{d}{dt} (\mathbf{T}^{-1} \boldsymbol{\psi}_{dq0}) = \mathbf{T}^{-1} \frac{d\boldsymbol{\psi}_{dq0}}{dt} + \frac{d\mathbf{T}^{-1}}{dt} \boldsymbol{\psi}_{dq0}$:

\begin{equation}
    \mathbf{v}_{dq0} = \mathbf{R} \mathbf{i}_{dq0} + \frac{d\boldsymbol{\psi}_{dq0}}{dt} + \mathbf{T} \frac{d\mathbf{T}^{-1}}{dt} \boldsymbol{\psi}_{dq0}
    \label{eq:vdq_flux}
\end{equation}

O termo $\mathbf{T} \frac{d\mathbf{T}^{-1}}{dt}$ representa o acoplamento entre os eixos devido à velocidade angular elétrica $\omega_e$:

\begin{equation}
    \mathbf{T} \frac{d\mathbf{T}^{-1}}{dt} = \omega_e \begin{bmatrix} 0 & -1 & 0 \\ 1 & 0 & 0 \\ 0 & 0 & 0 \end{bmatrix}
\end{equation}

Em seguida, aplica-se a transformação na equação de fluxo (2):

\begin{equation}
    \boldsymbol{\psi}_{dq0} = \mathbf{T} \boldsymbol{\psi}_{abc} = \mathbf{T} (\mathbf{L}_{abc} \mathbf{i}_{abc} + \boldsymbol{\psi}_{m,abc})
\end{equation}

\begin{equation}
    \boldsymbol{\psi}_{dq0} = \mathbf{T} \mathbf{L}_{abc} \mathbf{T}^{-1} \mathbf{i}_{dq0} + \mathbf{T} \boldsymbol{\psi}_{m,abc}
\end{equation}

A matriz de indutância no referencial $dq0$, $\mathbf{L}_{dq0}$, é obtida pela transformação de similaridade $\mathbf{T} \mathbf{L}_{abc} \mathbf{T}^{-1}$. Esta operação desacopla as fases e elimina a dependência da posição do rotor, resultando em uma matriz diagonal constante:

\begin{equation}
    \mathbf{L}_{dq0} = \begin{bmatrix} L_d & 0 & 0 \\ 0 & L_q & 0 \\ 0 & 0 & L_{al} \end{bmatrix}
\end{equation}

onde as indutâncias de eixo direto e quadratura são constantes, e dadas por:

\begin{equation}
    L_d = L_{al} + \frac{3}{2}(L_{aa0} + L_{g2})
\end{equation}
\begin{equation}
    L_q = L_{al} + \frac{3}{2}(L_{aa0} - L_{g2})
\end{equation}

Definindo o fluxo dos ímãs transformado como $\boldsymbol{\psi}_{m,dq0} = \mathbf{T} \boldsymbol{\psi}_{m,abc}$, a equação de fluxo no referencial $dq0$ é dada por:

\begin{equation}
    \boldsymbol{\psi}_{dq0} = \mathbf{L}_{dq0} \mathbf{i}_{dq0} + \boldsymbol{\psi}_{m,dq0}
    \label{eq:flux_dq}
\end{equation}

Substituindo \eqref{eq:flux_dq} em \eqref{eq:vdq_flux}:

\begin{equation}
    \mathbf{v}_{dq0} = \mathbf{R} \mathbf{i}_{dq0} + \mathbf{L}_{dq0} \frac{d\mathbf{i}_{dq0}}{dt} + \mathbf{T} \frac{d\mathbf{T}^{-1}}{dt} \mathbf{L}_{dq0} \mathbf{i}_{dq0} + \mathbf{e}_{dq0}
\end{equation}

onde $\mathbf{e}_{dq0}$ é a força contra-eletromotriz no referencial $dq0$, dada por:

\begin{equation}
    \mathbf{e}_{dq0} = \frac{d\boldsymbol{\psi}_{m,dq0}}{dt} + \mathbf{T} \frac{d\mathbf{T}^{-1}}{dt} \boldsymbol{\psi}_{m,dq0}
\end{equation}

A forma matricial explícita no referencial $dq$ torna-se:

\begin{equation}
    \frac{d\mathbf{i}_{dq0}}{dt} = \mathbf{A} \mathbf{i}_{dq0} + \mathbf{B} (\mathbf{v}_{dq0} - \mathbf{e}_{dq0})
\end{equation}

onde as matrizes de estado $\mathbf{A}$ e de entrada $\mathbf{B}$ são dadas por:

\begin{equation}
    \mathbf{A} = \begin{bmatrix} -\frac{R_s}{L_d} & \omega_e \frac{L_q}{L_d} \\ -\omega_e \frac{L_d}{L_q} & -\frac{R_s}{L_q} \end{bmatrix}, \quad
    \mathbf{B} = \begin{bmatrix} \frac{1}{L_d} & 0 \\ 0 & \frac{1}{L_q} \end{bmatrix}
\end{equation}

Isolando as derivadas para cada componente:

\begin{equation}
    \frac{dI_d}{dt} = \frac{1}{L_d} (V_d - R_s I_d + \omega_e L_q I_q - e_d)
\end{equation}

\begin{equation}
    \frac{dI_q}{dt} = \frac{1}{L_q} (V_q - R_s I_q - \omega_e L_d I_d - e_q)
\end{equation}

\begin{figure}[htbp]
\centerline{\includegraphics[width=\columnwidth]{back_emf_plot.png}}
\caption{Formas de onda da força contra-eletromotriz trapezoidal no referencial trifásico ($e_a, e_b, e_c$) e suas componentes no referencial síncrono ($e_d, e_q$).}
\label{fig:back_emf}
\end{figure}

A Fig. \ref{fig:back_emf} ilustra as formas de onda trapezoidais das fases e o resultado da transformação para o referencial dq. Nota-se que $e_q$ apresenta ondulações características (harmônicos de ordem $6k$) em vez de ser um valor DC puro como no PMSM senoidal, e $e_d$ também apresenta ondulações em torno de zero.

\subsection{Modelo Matemático do PMSM}
Diferente do BLDC, o Motor Síncrono de Ímãs Permanentes (PMSM) possui uma distribuição de fluxo senoidal. Neste caso, no referencial síncrono $dq$, as componentes da força contra-eletromotriz tornam-se constantes:

\begin{equation}
    e_d = 0
\end{equation}

\begin{equation}
    e_q = \omega_e \lambda_m
\end{equation}

Substituindo estas definições nas equações (9) e (10), obtemos as equações de estado para o PMSM:

\begin{equation}
    \frac{dI_d}{dt} = \frac{1}{L_d} (V_d - R_s I_d + \omega_e L_q I_q)
\end{equation}

\begin{equation}
    \frac{dI_q}{dt} = \frac{1}{L_q} (V_q - R_s I_q - \omega_e L_d I_d - \omega_e \lambda_m)
\end{equation}

\subsection{Equação de Torque}
A produção de torque eletromagnético em máquinas síncronas pode ser descrita de forma geral pelo balanço de potência no referencial $dq$. O torque é dado por:

\begin{equation}
    T_e = \frac{3}{2} P \frac{e_d I_d + e_q I_q}{\omega_e} + T_{relutancia}
\end{equation}

Para máquinas de ímãs permanentes montados na superfície (SPMSM), $L_d = L_q$, o que elimina o termo de torque de relutância.

No caso do PMSM com fluxo senoidal, onde $e_d = 0$ e $e_q = \omega_e \lambda_m$, a equação simplifica-se para a forma clássica:

\begin{equation}
    T_e = \frac{3}{2} P \lambda_m I_q
\end{equation}

Para o motor BLDC, como visto anteriormente, $e_d$ e $e_q$ variam com a posição do rotor, resultando em ondulações de torque se as correntes $I_d$ e $I_q$ forem mantidas constantes. A simulação utiliza a forma geral baseada nas componentes de força contra-eletromotriz para capturar este comportamento.

A dinâmica mecânica é descrita por:

\begin{equation}
    J \frac{d\omega_m}{dt} = T_e - T_L - B \omega_m - T_c
    \label{eq:mech}
\end{equation}

onde $J$ é o momento de inércia, $\omega_m$ é a velocidade mecânica, $T_L$ é o torque de carga, $B$ é o coeficiente de atrito viscoso, e $T_c$ é o torque de atrito de Coulomb.

\subsection{Modelo do Inversor}
O Inversor de Fonte de Tensão (VSI) trifásico é modelado idealmente, assumindo que as tensões de referência geradas pelo controlador são aplicadas com precisão aos terminais do motor, limitadas apenas pela tensão do barramento CC $V_{bus}$. Os limites da Modulação por Largura de Pulso Vetorial Espacial (SVPWM) são considerados saturando a magnitude do vetor de tensão para $V_{bus}/\sqrt{3}$.

\section{Estratégia de Controle}
A estratégia de controle adotada neste trabalho baseia-se na técnica de Controle Orientado a Campo (FOC), conforme detalhado em \cite{b3}. A estrutura de controle utiliza malhas em cascata, empregando controladores do tipo Proporcional-Integral (PI) para a regulação das correntes de eixo direto e quadratura, bem como para a velocidade do rotor.

\subsection{Projeto dos Controladores de Corrente}
O projeto dos controladores de corrente segue uma metodologia de alocação de polos baseada na resposta em frequência desejada. A função de transferência de malha fechada é aproximada para um sistema de segunda ordem, permitindo relacionar os ganhos do controlador ($k_{p}, k_{i}$) com a frequência natural ($\omega_n$) e o fator de amortecimento ($\xi$).

A banda passante ($\omega_b$) da malha de corrente é projetada para ser aproximadamente 10 vezes maior que a da malha de velocidade, garantindo o desacoplamento dinâmico. Neste trabalho, adotou-se $\omega_b = 350$ Hz e $\xi = 4$. Os ganhos resultantes são calculados conforme as equações analíticas apresentadas em \cite{b3}:

\begin{equation}
    k_{p_{iq}} = \frac{\xi \omega_b 2 L_q}{\sqrt{2\xi^2 + 1 + \sqrt{(1+2\xi)^2+1}}}
\end{equation}

\begin{equation}
    k_{i_{iq}} = \frac{L_q \omega_b^2}{2\xi^2 + 1 + \sqrt{(1+2\xi)^2+1}}
\end{equation}

Substituindo os parâmetros, obtém-se $k_{p} = 119$ e $k_{i} = 4015$. O acoplamento entre os eixos $d$ e $q$ é tratado como um distúrbio a ser compensado pela ação integral do controlador. A estrutura completa do controle FOC implementado é ilustrada na Fig. \ref{fig:foc_diagram}.

\begin{figure}[htbp]
\centerline{\includegraphics[width=\columnwidth]{foc_diagram.png}}
\caption{Diagrama de blocos do controle FOC \cite{b3}.}
\label{fig:foc_diagram}
\end{figure}

\subsection{Projeto do Controlador de Velocidade}
O controlador de velocidade é projetado assumindo que a dinâmica da malha de corrente é suficientemente rápida para ser desprezada. A banda passante escolhida foi $\omega_b = 35$ Hz com um fator de amortecimento $\xi = 1.0$. Os ganhos obtidos para o controlador de velocidade são $k_{p\omega} = 1.25$ e $k_{i\omega} = 55$.

A saída do controlador de velocidade (referência de corrente $I_q^*$) é saturada em $I_{max} = 8$ A para proteger o motor e evitar a desmagnetização dos ímãs permanentes. A referência de corrente de eixo direto é mantida em zero ($I_d^* = 0$).

\section{Resultados da Simulação}
A simulação foi realizada utilizando Python. Os parâmetros do motor utilizados são: $P=21$, $R_s=4.485 \Omega$, $L_d=L_q=54.8$ mH, $\lambda_m=0.201$ Wb, $J=0.1444$ kg$\cdot$m$^2$, $B=0.0057$ Nms/rad.

O perfil de simulação consiste em:
\begin{itemize}
    \item $t=0.0s$: Início em 40 RPM.
    \item $t=0.2s$: Degrau de torque de carga de 20 Nm aplicado.
    \item $t=0.4s$: Degrau de referência de velocidade para 80 RPM.
    \item $t=0.6s$: Degrau de referência de velocidade de volta para 40 RPM.
    \item $t=0.8s$: Torque de carga removido.
\end{itemize}

A Fig. \ref{fig:results} mostra a resposta do sistema.

\begin{figure}[htbp]
\centerline{\includegraphics[width=\columnwidth]{sim_results.png}}
\caption{Resultados da simulação mostrando Velocidade, Torque e Correntes ($I_d, I_q$).}
\label{fig:results}
\end{figure}

O controlador de velocidade rastreia a referência de RPM com precisão e mínimo sobressinal. Quando o torque de carga é aplicado em $t=0.2s$, observa-se uma pequena queda de velocidade, que é rapidamente rejeitada pelo controlador à medida que $I_q$ aumenta para gerar o torque eletromagnético necessário. A corrente $I_d$ é mantida em zero, garantindo uma operação eficiente.

\section{Conclusão}
Uma simulação completa de um acionamento PMSM usando FOC foi apresentada. A implementação modular em Python permite testes fáceis de diferentes parâmetros de controle e características do motor. Os resultados confirmam a robustez do esquema FOC em lidar com distúrbios de carga e rastrear referências de velocidade.

\appendices
\section{Código da Simulação}
Os códigos fonte da simulação desenvolvida em Python são apresentados a seguir.

% \subsection{PMSMMotor.py}
% \lstinputlisting{../Sim/PMSMMotor.py}

% \subsection{FOCController.py}
% \lstinputlisting{../Sim/FOCController.py}

% \subsection{Inverter.py}
% \lstinputlisting{../Sim/Inverter.py}

% \subsection{Sensors.py}
% \lstinputlisting{../Sim/Sensors.py}

% \subsection{Simulate.py}
% \lstinputlisting{../Sim/Simulate.py}

\begin{thebibliography}{00}
\bibitem{b1} F. Blaschke, "The principle of field orientation as applied to the new TRANSVECTOR closed-loop control system for rotating-field machines," Siemens Review, vol. 39, no. 5, pp. 217-220, 1972.
\bibitem{b2} MathWorks, "BLDC - Three-winding brushless direct current motor with trapezoidal flux distribution," [Online]. Available: https://www.mathworks.com/help/sps/ref/bldc.html.
\bibitem{b3} Matheus Alexandre Bevilaqua, "Implementação do Controle de Velocidade de Motores Síncronos a Ímãs Permanentes em Plataforma LabVIEW FPGA", Universidade do Estado de Santa Catarina, 2015.
\end{thebibliography}

\end{document}
