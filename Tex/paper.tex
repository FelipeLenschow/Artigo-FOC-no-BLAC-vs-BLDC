\documentclass[conference]{IEEEtran}
\IEEEoverridecommandlockouts
% The preceding line is only needed to identify funding in the first footnote. If that is unneeded, please comment it out.
\usepackage{cite}
\usepackage{amsmath,amssymb,amsfonts}
\usepackage{algorithmic}
\usepackage{graphicx}
\usepackage{textcomp}
\usepackage{listings}
\usepackage{xcolor}

\lstset{
    language=Python,
    basicstyle=\ttfamily\footnotesize,
    keywordstyle=\color{blue},
    stringstyle=\color{red},
    commentstyle=\color{green!60!black},
    numbers=left,
    numberstyle=\tiny,
    stepnumber=1,
    numbersep=5pt,
    backgroundcolor=\color{white},
    showspaces=false,
    showstringspaces=false,
    showtabs=false,
    frame=single,
    tabsize=2,
    captionpos=b,
    breaklines=true,
    breakatwhitespace=false,
    escapeinside={\%*}{*)}
}

\begin{document}

\title{Simulação e Controle de um Motor Síncrono de Ímãs Permanentes usando Controle Orientado a Campo\\
}

\author{\IEEEauthorblockN{1\textsuperscript{st} Felipe Lenschow}
\IEEEauthorblockA{\textit{Programa de pós graduação em engenharia elétrica} \\
\textit{Universidade do Estado de Santa Catarina}\\
Joinville, Santa Catarina, Brasil \\
felipe.lenschow@edu.udesc.br}
}

\maketitle

\begin{abstract}
Este artigo apresenta a modelagem e simulação de um sistema de acionamento de Motor Síncrono de Ímãs Permanentes (PMSM) utilizando Controle Orientado a Campo (FOC). O modelo matemático do PMSM no referencial dq é derivado, e uma estratégia de controle empregando controladores Proporcional-Integral (PI) para regulação de velocidade e corrente é implementada. A simulação é desenvolvida em Python, permitindo uma análise modular e flexível do comportamento dinâmico do motor sob condições variadas de carga e velocidade. Os resultados demonstram a eficácia da estratégia FOC em manter um controle preciso de velocidade e geração eficiente de torque.
\end{abstract}

\begin{IEEEkeywords}
PMSM, Controle Orientado a Campo, Simulação, Python, Acionamento de Motor
\end{IEEEkeywords}

\section{Introdução}
Motores Síncronos de Ímãs Permanentes (PMSMs) são amplamente utilizados em aplicações industriais, veículos elétricos e robótica devido à sua alta eficiência, alta densidade de potência e excelente desempenho dinâmico. Para alcançar um controle de alto desempenho, o Controle Orientado a Campo (FOC) é comumente empregado. O FOC permite o controle independente de fluxo e torque transformando as correntes trifásicas do estator para um referencial girante (referencial dq) alinhado com o fluxo do rotor \cite{b1}.

Este artigo detalha o desenvolvimento de um ambiente de simulação para um sistema de acionamento PMSM. A simulação inclui a física do motor, o inversor de fonte de tensão e o algoritmo FOC. O objetivo é fornecer uma compreensão clara da dinâmica do sistema e validar a estratégia de controle através de simulação numérica.

\section{Modelo do Sistema}

\subsection{Modelo Matemático do BLDC}
O modelo dinâmico do motor BLDC pode ser derivado a partir das equações de tensão de fase. Conforme descrito em \cite{b2}, as tensões nos enrolamentos do estator são definidas por:

\begin{equation}
    \begin{bmatrix} v_a \\ v_b \\ v_c \end{bmatrix} = 
    \begin{bmatrix} R_s & 0 & 0 \\ 0 & R_s & 0 \\ 0 & 0 & R_s \end{bmatrix} 
    \begin{bmatrix} i_a \\ i_b \\ i_c \end{bmatrix} + 
    \frac{d}{dt} \begin{bmatrix} \psi_a \\ \psi_b \\ \psi_c \end{bmatrix}
    \label{eq:vabc}
\end{equation}

onde $\psi$ representa o fluxo total concatenado em cada enrolamento, dado por:

\begin{equation}
    \begin{bmatrix} \psi_a \\ \psi_b \\ \psi_c \end{bmatrix} = 
    \mathbf{L}_{abc} \begin{bmatrix} i_a \\ i_b \\ i_c \end{bmatrix} + 
    \begin{bmatrix} \psi_{am} \\ \psi_{bm} \\ \psi_{cm} \end{bmatrix}
\end{equation}

Assumindo um estator simétrico e equilibrado onde a indutância própria é $L_s$ e a mútua é $M_s$, e negligenciando a variação da indutância com a posição do rotor ($L_m = 0$), a matriz de indutância $\mathbf{L}_{abc}$ é constante:

\begin{equation}
    \mathbf{L}_{abc} = \begin{bmatrix} L_s & M_s & M_s \\ M_s & L_s & M_s \\ M_s & M_s & L_s \end{bmatrix}
\end{equation}

Para simplificar a análise e o controle, aplica-se a Transformada de Park para converter as variáveis do referencial trifásico ($abc$) para o referencial síncrono girante ($dq0$). A transformação é definida por $\mathbf{x}_{dq0} = \mathbf{T} \mathbf{x}_{abc}$, onde $\mathbf{T}$ é a matriz de transformação.

Substituindo $\mathbf{x}_{abc} = \mathbf{T}^{-1} \mathbf{x}_{dq0}$ na equação de tensão \eqref{eq:vabc}:

\begin{equation}
    \mathbf{T}^{-1} \mathbf{v}_{dq0} = \mathbf{R} \mathbf{T}^{-1} \mathbf{i}_{dq0} + \mathbf{L}_{abc} \frac{d}{dt} (\mathbf{T}^{-1} \mathbf{i}_{dq0}) + \mathbf{e}_{abc}
\end{equation}

Multiplicando ambos os lados por $\mathbf{T}$:

\begin{equation}
    \mathbf{v}_{dq0} = \mathbf{R} \mathbf{i}_{dq0} + \mathbf{T} \mathbf{L}_{abc} \frac{d}{dt} (\mathbf{T}^{-1} \mathbf{i}_{dq0}) + \mathbf{T} \mathbf{e}_{abc}
\end{equation}

Expandindo a derivada do produto $\frac{d}{dt} (\mathbf{T}^{-1} \mathbf{i}_{dq0}) = \mathbf{T}^{-1} \frac{d\mathbf{i}_{dq0}}{dt} + \frac{d\mathbf{T}^{-1}}{dt} \mathbf{i}_{dq0}$:

\begin{equation}
    \mathbf{v}_{dq0} = \mathbf{R} \mathbf{i}_{dq0} + \mathbf{T} \mathbf{L}_{abc} \mathbf{T}^{-1} \frac{d\mathbf{i}_{dq0}}{dt} + \mathbf{T} \mathbf{L}_{abc} \frac{d\mathbf{T}^{-1}}{dt} \mathbf{i}_{dq0} + \mathbf{e}_{dq0}
\end{equation}

A matriz de indutância no referencial dq é diagonal para uma máquina de polos lisos, com $L_d = L_q = L_s - M_s$. O termo $\mathbf{T} \mathbf{L}_{abc} \frac{d\mathbf{T}^{-1}}{dt}$ resulta nas tensões de acoplamento devido à velocidade.

Finalmente, as equações de estado para as correntes $I_d$ e $I_q$ são obtidas isolando as derivadas:

\begin{equation}
    \frac{dI_d}{dt} = \frac{1}{L_d} (V_d - R_s I_d + \omega_e L_q I_q)
\end{equation}

\begin{equation}
    \frac{dI_q}{dt} = \frac{1}{L_q} (V_q - R_s I_q - \omega_e L_d I_d - \omega_e \lambda_m)
\end{equation}

Note que para o BLDC com fluxo trapezoidal, o termo de força contra-eletromotriz $\mathbf{e}_{dq0}$ conteria harmônicos, mas para fins de controle FOC fundamental, aproxima-se para o modelo senoidal acima.

O torque eletromagnético é dado por:

\begin{equation}
    T_e = \frac{3}{2} P (\lambda_m I_q + (L_d - L_q) I_d I_q)
\end{equation}

A dinâmica mecânica é descrita por:

\begin{equation}
    J \frac{d\omega_m}{dt} = T_e - T_L - B \omega_m - T_c
    \label{eq:mech}
\end{equation}

onde $J$ é o momento de inércia, $\omega_m$ é a velocidade mecânica, $T_L$ é o torque de carga, $B$ é o coeficiente de atrito viscoso, e $T_c$ é o torque de atrito de Coulomb.

\subsection{Modelo do Inversor}
O Inversor de Fonte de Tensão (VSI) trifásico é modelado idealmente, assumindo que as tensões de referência geradas pelo controlador são aplicadas com precisão aos terminais do motor, limitadas apenas pela tensão do barramento CC $V_{bus}$. Os limites da Modulação por Largura de Pulso Vetorial Espacial (SVPWM) são considerados saturando a magnitude do vetor de tensão para $V_{bus}/\sqrt{3}$.

\section{Estratégia de Controle}
A estratégia FOC é implementada com uma estrutura de controle em cascata.

\subsection{Malha de Controle de Corrente}
Dois controladores PI internos regulam as correntes $I_d$ e $I_q$. A referência de $I_d$ é definida como zero ($I_d^* = 0$) para maximizar o torque por ampère para o modelo SPMSM. A referência de $I_q$ é fornecida pela malha externa de velocidade. Termos de desacoplamento são calculados para compensar os efeitos de acoplamento cruzado entre os eixos d e q mostrados em \eqref{eq:vd} e \eqref{eq:vq}.

\subsection{Malha de Controle de Velocidade}
Um controlador PI externo regula a velocidade do motor. O erro entre a velocidade de referência $\omega_{ref}$ e a velocidade medida $\omega_m$ aciona o controlador PI para gerar a corrente de torque de referência $I_q^*$. A saída do controlador de velocidade é saturada para limitar a corrente máxima e proteger o motor e o inversor.

\section{Resultados da Simulação}
A simulação foi realizada utilizando Python. Os parâmetros do motor utilizados são: $P=21$, $R_s=4.485 \Omega$, $L_d=L_q=54.8$ mH, $\lambda_m=0.201$ Wb, $J=0.1444$ kg$\cdot$m$^2$, $B=0.0057$ Nms/rad.

O perfil de simulação consiste em:
\begin{itemize}
    \item $t=0.0s$: Início em 40 RPM.
    \item $t=0.2s$: Degrau de torque de carga de 20 Nm aplicado.
    \item $t=0.4s$: Degrau de referência de velocidade para 80 RPM.
    \item $t=0.6s$: Degrau de referência de velocidade de volta para 40 RPM.
    \item $t=0.8s$: Torque de carga removido.
\end{itemize}

A Fig. \ref{fig:results} mostra a resposta do sistema.

\begin{figure}[htbp]
\centerline{\includegraphics[width=\columnwidth]{sim_results.png}}
\caption{Resultados da simulação mostrando Velocidade, Torque e Correntes ($I_d, I_q$).}
\label{fig:results}
\end{figure}

O controlador de velocidade rastreia a referência de RPM com precisão e mínimo sobressinal. Quando o torque de carga é aplicado em $t=0.2s$, observa-se uma pequena queda de velocidade, que é rapidamente rejeitada pelo controlador à medida que $I_q$ aumenta para gerar o torque eletromagnético necessário. A corrente $I_d$ é mantida em zero, garantindo uma operação eficiente.

\section{Conclusão}
Uma simulação completa de um acionamento PMSM usando FOC foi apresentada. A implementação modular em Python permite testes fáceis de diferentes parâmetros de controle e características do motor. Os resultados confirmam a robustez do esquema FOC em lidar com distúrbios de carga e rastrear referências de velocidade.

\appendices
\section{Código da Simulação}
Os códigos fonte da simulação desenvolvida em Python são apresentados a seguir.

\subsection{PMSMMotor.py}
\lstinputlisting{../Sim/PMSMMotor.py}

\subsection{FOCController.py}
\lstinputlisting{../Sim/FOCController.py}

\subsection{Inverter.py}
\lstinputlisting{../Sim/Inverter.py}

\subsection{Sensors.py}
\lstinputlisting{../Sim/Sensors.py}

\subsection{Simulate.py}
\lstinputlisting{../Sim/Simulate.py}

\begin{thebibliography}{00}
\bibitem{b1} F. Blaschke, "The principle of field orientation as applied to the new TRANSVECTOR closed-loop control system for rotating-field machines," Siemens Review, vol. 39, no. 5, pp. 217-220, 1972.
\bibitem{b2} MathWorks, "BLDC - Three-winding brushless direct current motor with trapezoidal flux distribution," [Online]. Available: https://www.mathworks.com/help/sps/ref/bldc.html.
\end{thebibliography}

\end{document}
